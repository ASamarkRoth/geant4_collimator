\documentclass[a4paper]{article}

\usepackage[]{amsmath}
\begin{document}

\section{Collimator simulations for development of a scanning system}

The \textit{Compex} project comprises the development and study of a novel HPGe detector set-up. In its' beauty it will consists of five \textit{Compex}<++> Clovers, i.e. five groups of four crystals brought together within a capsule.
In nuclear spectroscopy experiments where very exotic structures are to be studied it is necessary to employ a high resolution HPGe detectors. In addition these detectors need to be compact. The compact and exotic together brings \textit{Compex}. In order to confidently discover new physics and propose new nuclear structure properties it is of utmost importance to know the characteristics of the detectors. In this regard it is hence essential to characterise the \textit{Compex} set-up in detail.

The characterisation of HPGe crystals is achieved with a scanning system. There exists some different scanning systems \cite{}. All systems utilise a collimator in some way. In the traditional scanning system a $\gamma$-beam is directed to one end of the crystal in a right angle. Surrounded on the sides of the crystal other $\gamma$-detectors, such as scintillators, are placed. With the help of slits $90 ^\circ$ Compton scattered $\gamma$-rays with a specific energy can be gated on. The origin of such signals can be narrowed down to a small region in the crystal and therefore its properties here can be known. By doing this for many regions one scans a complete crystal and studies:
\begin{itemize}
  \item Energy resolution
  \item Signal amplitude
  \item Rise time
\end{itemize}

Other scanning techniques are the pulse shape comparison scan and the mix of PSCS and $\gamma$-ray imaging.
A first intention is to characterise the \textit{Compex} crystals with the traditional technique.
To be able to do this one needs a collimator.
Important properties of the collimator are:
\begin{itemize}
  \item Divergence of $\gamma$-rays
  \item Scattering efficiency (full energy efficiency)
\end{itemize}
The simplest collimator is a volume with a cylinder hole where the volume can absorb $\gamma$-rays which are emitted badly.
By adjusting the diameter of the cylinder the beam divergence can easily be determined.
The scattering efficiency is not trivial to know.
The idea of integrating cones on top of the cylinder was told to be a preferable configuration concerning the scattering efficiency.
In this small project the performance of different collimators have been studied to determine the suitable collimator in use within the \textit{Compex} project.

\section{Geometry definition}

\section{Method}

\section{Results}

\section{Conclusion}



\end{document}
